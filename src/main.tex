% main.tex
\documentclass[twocolumn]{jarticle}
\usepackage[top=30truemm,bottom=25truemm,left=20truemm,right=20truemm]{geometry}
\usepackage[dvipdfmx]{graphicx,color}
\usepackage{cite}
\usepackage{array}

\pagestyle{empty}
\setlength{\columnsep}{7mm}

\setlength{\textfloatsep}{6pt plus 2pt minus 2pt}
\setlength{\intextsep}{6pt plus 2pt minus 2pt}
\setlength{\abovecaptionskip}{2pt}
\setlength{\belowcaptionskip}{2pt}

\title{\large\gt マルチモーダルな感覚刺激に基づく加速度感覚誘発システム}

\author{\normalsize
    \begin{tabular}{c}
        高須賀 颯太$^{1}$ \quad 竹中 健介$^{1}$ \quad 島 圭介$^{2}$ \quad 島谷 康司$^{2}$\\
        \vspace{1mm}
        $^{1}$横浜国立大学大学院環境情報学府 \quad $^{2}$広島県立大学保健福祉学科
    \end{tabular}
}
\date{}

\begin{document}
    \maketitle

    \section{はじめに}
    HMD型VRは訓練・設計支援・教育など多様な領域へ普及している一方,利用継続を妨げる要因としてVR酔いが知られている\cite{Saredakis2020Factors}.VR酔いは,視覚運動により自己運動知覚が生じたにもかかわらず,前庭・体性感覚側の手掛かりが不足または不整合となることで生じやすい.その結果,吐き気やめまい,眼精疲労などの不快が発生し,体験の質を低下させる.また,没入感(presence)と不快はしばしば負に関連し,VR酔いの低減は体験価値の向上にも直結し得る\cite{Weech2019Presence}.

    低減策には視野制限等のソフトウェア手法があるが,本研究は不足している感覚手掛かりを付与する方針を取る.前庭系へ介入する手法として前庭電気刺激(Galvanic Vestibular Stimulation: GVS)が用いられ,VR体験における運動感覚の補助や感覚提示の拡張が報告されている\cite{Sra2019ProprioGVS}.また,頭部近傍への振動提示により歩行や慣性の手掛かりを低忠実度で付与し,VR酔い低減やリアリズム向上を示した研究がある\cite{Peng2020WalkingVibe}.さらに首筋の筋振動を用いて回転に伴うVR酔いを低減する手法も報告されており\cite{Song2025NeckVRrr},頭頸部への体性感覚提示は有望である.
    しかし,前庭・体性感覚のいずれか一方のみでは,視覚運動の多様な変化(加減速や速度変動)に対して十分な整合を確保できない可能性がある.そこで本研究は,視覚運動に同期したGVSと頬・首振動の同時提示により,加速度感覚の補完と感覚間整合の向上を狙う.

    \section{システム}
    VRアプリケーション(例:Unity)から,視点(あるいはユーザ座標系)における速度$v(t)$と加速度$a(t)$のみを取得し,刺激制御モジュールへ送信する.刺激制御モジュールはGVSドライバおよび振動アクチュエータ(頬・首)を駆動し,刺激ログ(時刻,刺激レベル,振動強度等)を保存する.

    \begin{figure}[t]
        \centering
        \fbox{\parbox{0.95\columnwidth}{\centering 図1:システム構成図(ダミー)\\
        速度・加速度取得 → 刺激生成 → GVS/振動提示}}
        \caption{提案システムの概略}
        \label{fig:system}
    \end{figure}

    刺激生成は速度および加速度から行う.加速度の符号(例:前後方向)に応じてGVSの極性を切り替え,絶対値に応じて強度レベルを選択する.振動刺激は速度・加速度の大きさに応じて強度(および必要に応じて周波数)を変調し,頬・首の提示部位の配分により運動方向の手掛かりを付与する.視覚と刺激の同期性を担保するため,刺激値の急峻な変化を抑えるフィルタと,安全のための飽和処理・緊急停止を実装する.

    \section{実験}
    被験者内比較として,前庭電気刺激は「なし」「弱(感覚閾値未満)」「強(約$2\,\mathrm{mA}$)」の3水準,振動刺激は「なし」「あり」の2水準とし,これらの組合せから6条件(case0--5)を設定した.case0は前庭刺激なしかつ振動なし,case1は弱い前庭刺激のみ,case2は強い前庭刺激のみ,case3は振動のみ,case4は弱い前庭刺激と振動の併用,case5は強い前庭刺激と振動の併用である.感覚閾値は被験者ごとに簡易手続きで同定し,弱条件は主観的に知覚されない(または極めて弱い)範囲で設定した.強条件は約$2\,\mathrm{mA}$を目安に設定した(上限・安全手順は実験プロトコルに従う).各条件は複数試行とし,条件順はカウンタバランスした.
    \begin{table}[t]
        \centering
        \footnotesize
        \caption{実験条件(case番号と刺激条件の対応)}
        \label{tab:conditions}
        \begin{tabular}{c>{\centering\arraybackslash}m{0.40\columnwidth}>{\centering\arraybackslash}m{0.22\columnwidth}}
            \hline
            case & 前庭電気刺激(GVS) & 振動刺激 \\
            \hline
            0 & 無 & 無 \\
            1 & 弱(感覚閾値未満) & 無 \\
            2 & 強(約$2\,\mathrm{mA}$) & 無 \\
            3 & 無 & 有 \\
            4 & 弱(感覚閾値未満) & 有 \\
            5 & 強(約$2\,\mathrm{mA}$) & 有 \\
            \hline
        \end{tabular}
    \end{table}
    VR酔いの程度はVRSQ\cite{Kim2018VRSQ}で評価し,没入感はVAS(Visual Analog Scale)で取得した.個人差を抑えるため,各条件のスコアからcase0(刺激なし)を基準とした差分を算出し,VRSQ\_diffおよびimmersive\_diffとして比較した.

    \begin{figure}[t]
        \centering
        \fbox{\parbox{0.95\columnwidth}{\centering 図2:VRSQ\_diff(ダミー)\\
        case0--5の箱ひげ図(後で差し替え)}}
        \caption{VRSQ\_diffの条件別比較(case0基準の差分)}
        \label{fig:vrsq}
    \end{figure}

    \begin{figure}[t]
        \centering
        \fbox{\parbox{0.95\columnwidth}{\centering 図3:immersive\_diff(ダミー)\\
        case0--5の箱ひげ図(後で差し替え)}}
        \caption{immersive\_diffの条件別比較(case0基準の差分)}
        \label{fig:imm}
    \end{figure}

    図\ref{fig:vrsq}にVRSQ\_diffを示す.VRSQ\_diffは負方向がVR酔いの低減を表す.前庭刺激のみ(case1, case2)および振動を含む条件(case3--5)のいずれも,case0と比較してVRSQ\_diffが有意に低下した(図中の有意差).また,弱い前庭刺激と振動を併用したcase4は,弱い前庭刺激のみのcase1および振動のみのcase3に対して有意に低下した.この結果は,弱い前庭刺激と振動提示の併用が,単独提示よりもVR酔い低減に寄与する可能性を示す.一方で,振動併用下における強度差(case4とcase5)については有意差が示されておらず,本実験条件では高電流化による上乗せ効果が明確ではないことが示唆される.

    図\ref{fig:imm}にimmersive\_diffを示す.immersive\_diffは正方向が没入感の向上を表す.強い前庭刺激のみのcase2はcase0に対して有意に上昇した.さらに,振動を含むcase3--5はいずれもcase0に対して有意に上昇した一方で,弱い前庭刺激のみのcase1では有意差が示されていない.以上より,没入感の向上には振動提示の寄与が大きく,前庭刺激は強度条件で補助的に寄与する可能性がある.

    \section{考察}
    VRSQ\_diffにおいて,前庭刺激のみ(case1, case2)と振動のみ(case3)のいずれでもVR酔いの低減が確認されたことから,前庭・体性感覚いずれの手掛かり付与も視覚運動との不一致を緩和し得る.特に,case4がcase1およびcase3よりも有意に低下したことは,弱い前庭刺激と振動提示の併用が相補的に作用し,視覚運動に対する整合性を高めた可能性を示す.先行研究でもGVSによる自己運動感覚の拡張\cite{Sra2019ProprioGVS}や頭頸部への振動提示によるVR酔い低減\cite{Peng2020WalkingVibe,Song2025NeckVRrr}が報告されており,本研究の結果はこれらの知見と整合する.

    一方で,振動併用下での前庭刺激強度差(case4とcase5)に明確な差が見られない点は,刺激強度の最適域や飽和の存在,あるいは個人差の影響を示唆する.加えて,case3では分散が大きく,一部の被験者で逆方向の変化も見られることから,振動の提示タイミングや強度変調が視覚運動と不整合になる場合に不快を増やす可能性がある.したがって,速度・加速度に対する同期誤差の定量化と,個人適応(閾値同定や強度スケーリング)を導入することが重要である.

    没入感(immersive\_diff)については,振動を含む条件で一貫して向上が見られたことから,頭頸部への体性感覚手掛かりが臨場感・身体感覚の補完に寄与したと考えられる.一方で,弱い前庭刺激のみでは有意な向上が見られないことから,感覚閾値未満の刺激は没入感向上よりも不一致低減側に寄与する設計となっている可能性がある.今後は,VR酔い低減と没入感向上の両立に向けて,刺激強度・部位・同期の設計空間を体系的に探索する.

    \section{終わりに}
    視覚運動(速度・加速度)に同期して前庭電気刺激と頬・首振動を同時制御する加速度感覚誘発システムを提案し,6条件の被験者内比較によりVRSQとVASを用いて評価した.その結果,いずれの刺激条件でもVR酔いの低減が示され,特に弱い前庭刺激と振動の併用は単独提示よりも低減効果が高い可能性が示唆された.今後は,同期誤差の定量化と個人適応を導入し,刺激設計の最適化を進める.

    \bibliographystyle{jplain}
    \bibliography{references}
\end{document}
